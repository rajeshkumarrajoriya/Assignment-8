\documentclass{article}
\usepackage[utf8]{inputenc}
\usepackage{karnaugh-map}

\title{Assignment 8}
\author{Rajesh Kumar Rajoriya}
\date{january 9th, 2021}

\begin{document}

\maketitle

\section{The Boolean Expression For e}
\begin{equation}
 $e = A.\overline{B}.\overline{C}.\overline{D} + A.B.\overline{C}.\overline{D} + \overline{A}.\overline{B}.C.\overline{D}+A.\overline{B}.C\overline{D}+A.B.C.\overline{D}+A.\overline{B}.\overline{C}.D$
\end{equation}

\section{K-Map}
Realisation using k-map when its POS.
\begin{figure}[h]
\centering
\begin{karnaugh-map}[4][4][1][][]
    \maxterms{0,1,4,5,7,9}
    \minterms{2,3,8,6}
    \autoterms[X]
    % note: position for start of \draw is (0, Y) where Y is
    % the Y size(number of cells high) in this case Y=2
  
    \implicant{0}{5}
    \implicant{5}{15}
    \implicant{1}{9}
    \draw[color=black, ultra thin] (0, 4) --
    node [pos=0.7, above right, anchor=south west] {$AB$} % YOU CAN CHANGE NAME OF VAR HERE, THE $X$ IS USED FOR ITALICS
    node [pos=0.7, below left, anchor=north east] {$CD$} % SAME FOR THIS
    ++(135:1);
\end{karnaugh-map}
\caption{K-Map for \textit{e} under given don't care condition.}
\end{figure}
\section{Simplified expression}
  e = (A+C).(A+\overline{B}).(\overline{B}+\overline{D})



\end{document}
